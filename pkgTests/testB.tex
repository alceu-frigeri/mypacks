% !TEX program = pdflatex
% !TEX ext =  --interaction=nonstopmode --enable-etex --enable-write18
% !BIB program = none
\documentclass[a4paper,10pt]{article}
\usepackage[times]{mypacks}
\usepackage[no alias,xtrakeys=H,xtraidx=x,undef color=green]{tikzquests}
\usepackage{tikzquads}
\usepackage{tikzdotncross}
%\usepackage{xcolor}
%\usepackage{EletronicaI}
\usepackage{CircuitosI}
%\usepackage{CircuitosII}
\usepackage{randomcircuits}
\usepackage{ifthen}
%\usetikzlibrary{knots}
\usetikzlibrary{intersections}


\newcommand{\XYZ}{some thing}
\defQuestion*{\XYZ}{"some question"}
\defQuestion{C.I:1a:QXx}{%
  \draw (0,0) \ncoord(ref) to[I,l=\Ia] ++(0,4) \ncoord(A) to[R=\Ra] ++(2,0) \ncoord(B)
  (B) to[V,l=\QuestVal{Beta}] ++(0,-2) \ncoord(C) node(nC){}
  (B) -- ++(2,0) \ncoord(Bb) to[R=\Rxa] ++(0,-2) \ncoord(Cc) -- (C) 
  ;
}[some comment]

\defQuestion{C.I:1a:QX}{%
  \draw (0,0) \ncoord(ref) to[I,l=\QuestVal{Ia}] ++(0,4) \ncoord(A) to[R=\QuestVal{Ra}] ++(2,0) \ncoord(B)
  (B) to[V,l=\QuestVal{Beta}] ++(0,-2) \ncoord(C) node(nC){}
  (B) -- ++(2,0) \ncoord(Bb) to[R=\QuestVal{Rxa}] ++(0,-2) \ncoord(Cc) -- (C) 
  ;
}[some remarks]


\defQuestion{sample 2}{%
\begin{knot}[draft mode=crossings,clip width=5]
		\strand
		(0,0) \ncoord(A) to[V,invert,l=\Vi] ++(0,3) \ncoord(V)
      to[R=\Ra] ++(2,0) 
      to[C] ++(2,0) \dotcoord(B) 
      -- ++(1,0) node[npn,anchor=B] (T1){}
		(A) -- (A -| B) \ncoord(Ba) to[R=\Rb] (B) to[R=\Rg] ++(0,3) \ncoord(C)
		(T1.E) to[R,l=\Rc] (T1.E |- A) -- (A)
		(T1.C) to[R,l_=\Rd] (T1.C |- C) -- (C -| A) -- ++(-2,0) \ncoord(X) to[V,l=\Vbc] (X |- A) -- (A)
    (T1.C) -- ++(1,0) node[ocirc]{} \ncoord(k) to[open,v=\Vo] (k |- A) node[ocirc]{} -- (A)
		 ;
\end{knot}
}


\defQuestion{sample}{%
		\draw
		(0,0) \ncoord(A) to[V,invert,l=\Vi] ++(0,3) \ncoord(V)
      to[R=\Ra] ++(2,0) 
      to[C] ++(2,0) \ncoord(B) 
      -- ++(1,0) node[pnp,anchor=B] (T1){}
		(A) -- (A -| B) \ncoord(Ba) to[R=\Rb] (B) to[R=\Rg] ++(0,3) \ncoord(C)
		(T1.C) to[R,l=\Rc] (T1.C |- A) -- (A)
		(T1.E) to[R,l_=\Rd] (T1.E |- C) -- (C -| A) -- ++(-2,0) \ncoord(X) to[V,l=\Vbc] (X |- A) -- (A)
    (T1.C) -- ++(1,0) node[ocirc]{} \ncoord(k) to[open,v=\Vo] (k |- A) node[ocirc]{} -- (A)
		 ;
}

\showcoordstrue
\begin{document}


\begin{tikzpicture}
  \path 
    (0,0) coordinate(X) 
    (1,1) coordinate(A) node[minimum size=4.3pt,inner sep=0pt](nA){}
    (2,2) coordinate(B) node[minimum size=4.3pt,inner sep=0pt](nB){}
    (3,3) coordinate(Y)
    ;
  \draw[rotate=45] 
    (A) +(-7pt/2,0) arc[start angle=180,end angle=0,radius=7pt/2]
    (B) +(-7pt/2,0) arc[start angle=180,end angle=0,radius=7pt/2]
    ;
    
    % Case (1)
  %\draw (X) -- (nA) -- (nB) -- (Y) ;
  
    % Case (2)
  %\foreach \x/\y in {X/nA,nA/nB,nB/Y} {\draw (\x) -- (\y);}

    % Case (3)
  \draw (X) \foreach \x in {nA,nB,Y} { -- (\x)};
  
  
\end{tikzpicture}


%\show\ncoord
\begin{tikzpicture}
      %% This is the reference, named, path
      %%
  		\draw[name path=base circ]
		(0,0) \ncoord(A) to[V,invert,l=$v_i(t)$] ++(0,2) -- ++(0,1)
      to[R=$R_i$] ++(2,0) 
      to[C] ++(1,0) \pincoord(B1) ++(1,0) \ncoord(B)
       ++(1,0) node[pnp,anchor=B] (T1){}
		(A) -- (A -| B)  to[R=$R_{b_2}$] ++(0,2) \ncoord(Bb) (B) ++(0,1) \ncoord(Cb) to[R=$R_{b_1}$] ++(0,2) \ncoord(C)
		(T1.C) to[R,l=$R_c$] (T1.C |- A) -- (A)
		(T1.E) to[R,l_=$R_e$] (T1.E |- C) -- (C -| A) -- ++(-2,0) \ncoord(X) to[V,l=$V_{cc}$] (X |- A) -- (A)
    (T1.C) -- ++(1,0) node[ocirc]{} \ncoord(k) to[open,v=$v_o(t)$] (k |- A) node[ocirc]{} -- (A)
    (Bb) -- (Cb);
    ;
    %% These are just a few, marked, coords (they could be part of the previous path
    %%
  \path (T1.E)  ++(1,0)      \pincoord(X1) -- ++(-10,2)     \pincoord(X2)
        (X1) ++(0,-1)        \dotpincoord(X1b) (X2) ++(0,-1)   \pincoord(X2b) 
        (T1.C)  ++(0.5,-0.5) \odotpincoord(X1a) (T1.C) ++(-9,0) \pincoord(X2a)
        (T1.B) \pincoord(B2,blue,225)
  ;
  %% And that's all, a few crossing lines
  %%
  \pathcross{B1}{T1.B}{base circ}[4pt]
  \pathcross*{X1}{X2}{base circ}[3pt]
  \pathcross[sec]{X2a}{X1a}{base circ}[6pt]
\end{tikzpicture}


%\tikzQuestion{sample}
%
%\tikzQuestion{sample 2}
%
%\defQuestion*{Testbeta}{
%supondo $\beta \approx \QuestVal{Beta}$ e \QuestVal{Ra} determine: \ftikzQuestion(0.25){C.I:1a:QX}
%}[that's some nice feature]
%
%
%test:\textQuestion{\XYZ}
%
%
%\textQuestion{Testbeta}[Beta={400}]
%
%\textQuestion{Testbeta}[Beta={\ensuremath{500_2^3}},Ra=100]
%
%\textQuestion{Testbeta}
%
%\ftikzQuestion(0.25){C.I:1a:QX}[Ra raw=\ensuremath{R_5},Rxa*=R_8]

\ftikzQuestion(0.25){C.I:ext:Q3}


\begin{equation}
\frac{u_{o_{_1}}}{u_{i_{_1}}} = \left( \left. \frac{u_{o_{_1}}}{u_{i_{_1}}} \right|_{u_{i_{_2}}=0} \right) * 
\frac{1 + \frac{\left. \frac{u_{o_{_2}}}{u_{i_{_2}}} \right|_{u_{o_{_1}}=0}}{-\left. \frac{u_{o_{_2}}}{u_{i_{_2}}} \right|_{ext}}}{1 + \frac{\left. \frac{u_{o_{_2}}}{u_{i_{_2}}} \right|_{u_{i_{_1}}=0}}{-\left. \frac{u_{o_{_2}}}{u_{i_{_2}}} \right|_{ext}} }
\end{equation}

\end{document}
